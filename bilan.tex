\subsection{L'informatique}
Ma passion pour l'informatique m'a fait me tourner vers un stage en lien avec le réseau et les systèmes d'informations. 
Ce stage fut pour moi l'occasion de découvrir un métier qui pourrait s'apparenter à Administrateur Réseau ou Administrateur système. Cela m'a permis de confirmer mon choix d'orientation. J'ai pris par exemple énormément de plaisir à sécuriser des serveurs des attaques extérieurs (changement de ports, cloisonnement, Fail2Ban\dots). Ce stage m'a permis de comprendre et de me rendre compte des techniques et méthodes utilisées dans un grand établissement comme le CNRS. En effet, dans ce type de grandes entreprises, la gestion du parc informatique, de son maintien et de ses mises à niveau nécessite une organisation et des méthodes qui s’avèrent complexes.

Par exemple, il ne suffit pas de brancher une machine à une prise réseau (RJ45) pour qu'elle soit connectée. En effet, afin de limiter les accès réseau et les risques d'intrusions informatiques, il faut au préalable demander l'enregistrement de la machine dans le serveur DHCP afin qu'elle soit connectée. Comprendre le fonctionnement du réseau informatique m'a vraiment plu. De plus, le réseau d'un établissement aussi important que le CNRS ne peut pas être parfait. Il y a toujours du travail afin d'améliorer l'expérience des utilisateurs (vitesse de connection, sécurité,\ldots).
\medbreak
Je me suis rendu compte que l'on ne peut pas faire ce que l'on veut sur les serveurs. En effet, on passe systématiquement par une phase de tests dans un environnement sûr afin de limiter les risques. Il faut alors comprendre le lien entre les différentes machines (lien réseau principalement), avant de reproduire une maquette de ces serveurs afin de tester ce que l'on veut faire. Cela nécessite une certaine rigueur car le bon fonctionnement de l'établissement dépend de nos actions.
\medbreak
En revanche, une partie qui me passionne moins dans l'informatique est le développement logiciel. En effet, les développements que j'ai eu à effectuer (PowerShell et shell) me sont apparus bien moins passionnant et intéressant, en particulier si le code était long. Un petit script de quelques lignes ne me posait pas problème, mais de longs projets me paraissaient moins attirants. 
\medbreak
Ce stage m'a permis de confirmer mon souhait d'orientation vers l'informatique. J'ai donc pu écarter la partie développement pour les raisons citées ci-dessus. Je pense donc choisir une filière tournée vers la cybersécurité ou le réseau. 
 
\subsection{Hotline et assistance}
J'ai trouvé très délicat d'échanger avec des personnes non initiées en informatique. En effet, les personnels qui venaient nous voir étaient souvent très énervées car elles avaient un problème qu'elles n'arrivaient pas à résoudre et ne pouvaient donc pas travailler. Il n'était pas rare que les employés pensaient que leur problème était de notre faute, bien qu'il arrivait rarement que ce soit le cas (mise à jour qui n'est pas bien passée, ou coupure réseau générale). Il a donc fallu gérer ces utilisateurs et chercher une solution à leur problème. Malgré tout, notre travail était toujours apprécié, surtout lorsque l'on réussissait à expliquer aux utilisateurs ce que l'on faisait. Les employés se sentaient donc directement concernés et acteurs dans ces moments.\\
La reconnaissance des personnels et la satisfaction liée au fait de réussir à aider une personne dans son travail sont toujours gratifiantes.

\subsection{Une entreprise pas uniquement tournée vers l'informatique}
Un aspect que j'ai beaucoup apprécié est le fait que cette entreprise n'est pas uniquement tournée vers l'informatique. Cela permet de découvrir de nouveaux domaines tout en faisant de l'informatique. En effet, sur le campus de Villejuif, il y avait des chercheurs en biologie, des médecins, des archivistes, des linguistes\dots \\
Cela permet de communiquer avec différents corps de métiers et de changer d'environnement. Cet aspect apporte une ouverture d'esprit et une certaine capacité d'adaptation qui est nécessaire afin de comprendre le problème qui est souvent lié à la profession de la personne (par exemple, le logiciel de la paye qui ne fonctionne pas normalement pour les agents de la paye, le driver qui ne s'installe pas pour les capteurs médicaux\dots). \\
Cela permet d'acquérir d'autres compétences, ainsi que de la culture générale.

\subsection{Bilan}
Ce stage fût pour moi l'occasion de me confronter au monde de l'entreprise. Bien que le stage de 3ème soit intéressant, il est bien trop court pour avoir un réel aperçu du monde de l'entreprise. Dans le cas de ce stage, même un mois m'a paru trop court. En effet, je n'ai pas disposé d'assez de temps pour réellement apprendre le métier et être utile à l'équipe. J'apportais des solutions, des idées parfois, mais j'avais l'impression de plutôt être un poids pour eux. Cependant, après une rapide intégration dans l'équipe, j'ai pu m'occuper des tâches qui m'étaient confiées avec plaisir, en mettant en pratique les compétences que j'ai acquises cette année à l'ESIEE, mais aussi celles que j'ai obtenue avec le Club*Nix.