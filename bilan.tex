\subsection{L'informatique}
Actuel trésorier du Club*Nix, club informatique de l'ESIEE, ma passion pour l'informatique m'a donc fait tourner vers un stage en lien avec le réseau et les systèmes d'informations. 
Ce stage fut pour moi l'occasion de découvrir un métier qui pourrait s'apparenter à Administrateur Réseau ou Administrateur système. Cela m'a permis de confirmer mon choix d'orientation. En effet, j'ai pris énormément de plaisir par exemple à sécuriser des serveurs des attaques extérieurs (changement de ports, cloisonnement, Fail2Ban\dots). Ce stage m'a permis de comprendre et de me rendre compte des techniques et méthodes utilisées dans une grande entreprise comme le CNRS. En effet, dans ce type de grandes entreprises, la moindre modification dans le parc informatique devient complexe. \textit{Par exemple, il ne suffit pas de connecter une machine à une prise réseau (RJ45) pour qu'elle soit connectée. En effet, afin de limiter les accès réseau et les risques d'intrusions informatique, il faut au préalable demandé l'enregistrement de la machine dans le serveur DHCP afin qu'elle soit connectée.} Comprendre le fonctionnement du réseau informatique m'a vraiment plu. De plus, une aussi grosse structure comme celle-la ne peut pas être parfaite. Il y a toujours du travail afin d'améliorer l'expérience des utilisateurs (vitesse de connections, sécurité,\ldots). 
\medbreak
De plus, je me suis rendu compte que l'on ne pouvait pas faire ce ue l'on voulait sur les serveur. En effet, on passait systématiquement par une phase de test dans un environnement sûr. Il fallait alors comprendre le lien entre les différentes machines (lien réseau principalement), avant de reproduire une maquette de ces serveurs afin de tester ce que l'on voulait faire. Cela démontre une certaine rigueur dans le domaine car en effet, on ne fait pas ce que l'on veut car nos actions et le bon fonctionnement de l'entreprise dépend de nos actions.
\medbreak
En revanche, une partie qui me passionne moins dans l'informatique est le développement logiciel. En effet, le peu de développement que j'ai eu à effectué (PowerShell et shell) m'est apparu bien moins passionnant et intéressant, en particulier si le code était long. Un petit script de quelques lignes ne me posait pas problèmes, mais de longs projets me paraissait moins attirant. 
\medbreak
Ce stage m'a permis de confirmer mon idée d'orientation qu'était l'informatique. J'ai donc pu éloigner la partie développement pour les raisons citées ci-dessus. Je pense donc m'orienter plus vers une filière cybersécurité ou réseau. 
 
\subsection{Hotline et assistance}
J'ai trouvé très difficile de traiter avec des personnes novices en informatique. En effet, les personnes qui venaient nous voir étaient souvent très énervées car elles avaient un problème et ne pouvaient pas travailler. Il n'était pas rare que les employés pensent que leur problème était de notre faute. Bien qu'il arrivait rarement que ce soit le cas (mise à jour qui n'est pas bien passée, ou coupure réseau générale), ce n'était en général pas de notre faute. Il a donc fallut gérer ces utilisateurs et chercher une solution à leur problème. Malgré tout, notre travail était toujours apprécié au final, d'autant plus que l'on essayait au maximum au personnel de le expliquer ce que l'on faisait. Elle se sentait donc directement concerné et actrice dans ces moments.\\
La reconnaissance des employés et la satisfaction liée au fait de réussir et d'aider une personne dans son travail, même si cela était involontaire de sa part, est toujours satisfaisante.

\subsection{Une entreprise pas uniquement tourné vers l'informatique}
Un aspect que j'ai beaucoup apprécié est le fait que cette entreprise n'est pas uniquement tourné vers l'informatique. Cela permet de découvrir de nouveaux domaines tout en faisant de l'informatique. En effet, sur le campus de Villejuif, il y avait des chercheurs en biologie, des médecins, des archivistes, des linguistes\dots \\
Cela permet de communiquer avec différents corps de métiers et de changer d'environnement. Cet aspect apporte une ouverture d'esprit et une une certaine capacité d'adaptation est nécessaire afin de comprendre le problème qui sont souvent liés à la profession de la personne (par exemple, le logiciel de la paye qui ne fonctionne pas normalement pour les agents de la paye, le driver qui ne s'installe pas pour les capteurs médicaux\dots). \\
Cela permet d'acquérir d'autres compétences, ainsi que de la culture générale.

\subsection{Bilan}
Ce stage fût pour moi l'occasion de me confronter une première fois au monde de l'entreprise. Bien que le stage de 3ème soit tout aussi formateur, il est bien trop court pour avoir un réel aperçu du monde de l'entreprise. Dans le cas de ce stage, même un mois m'a paru trop court. En effet, je n'ai pas disposé d'assez de temps pour réellement apprendre le métier et être utile à l'équipe. J'apportais des solutions, des idées parfois, mais j'avais l'impression de plutôt être un poids pour eux. 
\smallbreak
Cependant, après une rapide intégration dans l'équipe, j'ai pu m'occuper rapidement des taches qui m'étaient confiées avec plaisir, en mettant en pratique les compétences que j'ai acquises au long de cette année, mais aussi celles que j'ai obtenue avec le Club*Nix. 