Afin de clôturer la première année à l'ESIEE Paris, chaque élève doit effectuer un stage d'exécution d'une durée de 4 semaines en entreprise. Ce stage d’exécution permet à l’élève d’avoir un premier contact avec la vie active. 
J'ai choisis d'effectuer ce stage dans le domaine de l'informatique. En effet, cela correspond à mes envies actuelles d'orientation pour mes futures années à l'ESIEE. Ma mère travaillant au CNRS, j'ai pu obtenir le contact de Mr Etienne FAURE, responsable du pôle SI de l'IFSeM\footnotemark, pôle que je détaillerais plus tard. J'ai donc envoyé à Mr Faure une lettre de motivation et un Curriculum Vitae. A la suite d'un entretient téléphonique, il a accepté de me prendre comme stagiaire du 1er au 26 juillet 2019. 
\footnotetext{Ile-de-France Service Mutualisé}
L'IFSeM est situé au 7 rue Guy Moquet, à Villejuif. Je travaillais du lundi au vendredi, entre 9h et 17h42, avec donc une amplitude horaire de 7h42 par jours. Cette amplitude n'était pas fixe. En effet, certains jours on me demandait d'arriver plus tôt, ou plus tard. 
\medbreak
L'activité sur le campus n'était pas très importante. En effet, mon stage pendant le mois de juillet tombait sur une période durant laquelle de nombreux agents étaient en vacances. De plus, au vu des très fortes chaleur caniculaires qui ont eus lieu, bon nombre d'employés travaillaient depuis chez eux, ce qui ne m'étais pas permis. Sur le campus était présent une cantine. C'est un gros point fort pour la restauration.
\medbreak
Dans ce rapport, je parlerais dans un premier temps de l'entreprise, plus particulièrement du service où j'ai effectué mon stage : le SI de l'IFSeM. Ensuite, je parlerais du travail et des taches qui m'ont été confiées durant ce stage, puis du résultat que j'en ai tiré au point de vu personnel et professionnel. 