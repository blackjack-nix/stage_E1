Afin de clôturer la première année à l'ESIEE Paris, chaque élève doit effectuer un stage d'exécution d'une durée de 4 semaines en entreprise. Ce stage d’exécution permet à l’élève d’avoir un premier contact avec la vie active. 
J'ai choisi d'effectuer ce stage dans le domaine de l'informatique. En effet, cela correspond à mon projet professionnel et à mes souhaits d'orientation pour mes futures années à l'ESIEE. Ma mère travaillant au CNRS, j'ai pu obtenir le contact de Mr Etienne FAURE, responsable du pôle SI de l'IFSeM\footnotemark, pôle que je détaillerai plus tard. J'ai donc envoyé à Mr Faure une lettre de motivation et un Curriculum Vitae. A la suite d'un entretien téléphonique, il a accepté de me prendre comme stagiaire du 1er au 26 juillet 2019.
\footnotetext{Ile-de-France Service Mutualisé}
L'IFSeM est situé au 7 rue Guy Moquet, à Villejuif. Je travaillais du lundi au vendredi, entre 9h et 17h, avec une amplitude horaire de 7h par jour. Ces horaires n'étaient pas fixes. En effet, certains jours on me demandait d'arriver plus tôt, ou plus tard.
\medbreak
L'activité sur le campus du CNRS de Villejuif n'était pas très importante. En effet, mon stage pendant le mois de juillet tombait sur une période durant laquelle de nombreux agents étaient en congés. De plus, étant donné les très fortes chaleurs de ce mois de juillet, bon nombre d'employés travaillaient depuis chez eux, ce qui ne m'était pas permis. Il y avait un restaurant d'entreprise sur le campus.
\medbreak
Dans ce rapport, je parlerai dans un premier temps de l'entreprise, plus particulièrement du pôle où j'ai effectué mon stage : le SI de l'IFSeM. Ensuite, je parlerai du travail et des tâches qui m'ont été confiés durant ce stage, puis de l'enseignement que j'en ai tiré au point de vu personnel et professionnel.