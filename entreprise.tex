\subsection{Présentation de l'entreprise}
Le CNRS, ou Centre National de la Recherche Scientifique, est une institution de recherche parmi les plus importantes au monde. Pour relever les grands défis présents et à venir, ses scientifiques explorent le vivant, la matière, l’Univers et le fonctionnement des sociétés humaines. Mondialement reconnu pour l’excellence de ses travaux scientifiques, le CNRS est une référence, aussi bien dans l’univers de la recherche et du développement, que pour le grand public. Le CNRS étant une importante entreprise publique, les domaines de recherches sont très vastes (sciences, langues, histoires, mathématiques \ldots). 

\subsection{Présentation de l'IFSeM}
\textit{"Créé en juillet 2015, le Service Mutualisé d’Île-de-France, rattaché à la Délégation Paris-Villejuif, prend en charge des activités jusqu’alors exercées par les 5 délégations franciliennes et les regroupe en 4 pôles de compétences au bénéfice de toutes\footnotemark. "}
\footnotetext{https://www.dr1.cnrs.fr/spip.php?rubrique59}
\smallbreak
Voici ce que l'on peut lire sur le site de la délégation de Paris-Villejuif du CNRS. Comme expliqué en introduction, l'IFSeM est situé à Villejuif, sur le même campus que la DR01. L'IFSeM est constitué de 4 pôles : Formation, Achats, Système d'information et Patrimoine immobilier. Ces 4 pôles ont pour mission de "gagner en cohérence et permettre à toutes les délégations du territoire d’apporter le meilleur niveau de service à ses unités comme à son personnel".

\subsection{Présentation du pôle SI de l'IFSEM}
J'ai donc choisis, dans la continuité de mon projet à ESIEE Paris, d'effectuer ce stage d'un mois au service informatique de l'IFSeM (Ile-de-France Service Mutualisé), au pôle infrastructure. 
Ce pôle a pour objectif d'apporter une assistance technique à 4 délégations différentes du CNRS : la DR01, située à Villejuif, la DR02, à Paris-Centre, la DR04 à Gif-Sur-Yvettes et la DR05 située à Meudon. De plus, ce service à pour but de maintenir et mettre à jour les applications nationales du CNRS au travers de différents serveurs décentralisés. 
\newpage
\subsection{Fonctionnement de l'entreprise : Le travail en équipe}

Le service informatique de l'IFSeM est divisé en 3 sous-pôles (pour l'organigramme complet, se reporter, en annexe, au point 5.1.1) : 
\begin{itemize}
    \item Le support aux applications et offre de services : \textit{"support et exploitation de la téléphonie IP après déploiement, support des applications nationales du CNRS, création et accompagnement de la mise en oeuvre de l’offre de services nationale pour les délégations concernées et les laboratoires qui en dépendent.\footnotemark[3]"}
    \item La gestion des infrastructures : \textit{"gestion de parc informatique, sécurité informatique (communication/formation à destination des unités, support à la gestion des incidents), et gestion des systèmes et réseaux (support à l’ingénierie, gestion propre au Service Mutualisé, supervision des raccordements de sites).\footnotemark[3]"}
    \item Le développement logiciel : \textit{"besoins applicatifs propres au Service Mutualisé et applications métiers recouvrant les besoins communs aux 5 délégations.\footnotemark[3]"}
\end{itemize}
\footnotetext[3]{Voir note 2}
\smallbreak
Durant mon stage, j'ai été affilié au pôle "Gestion des infrastructures". Nous étions dans un bureau de 4 personnes. Cependant, la cision n'est pas totale entre ces 3 équipes. En effet, j'ai pu observer une réelle discussion et échanges entre chaque pôle, et chacun apportait son aide et participait aux solutions des autres groupes. 
Ces trois pôles sont sous la direction de Mr Etienne FAURE.


\subsubsection{Les réunions de service}
Lorsqu'un problème récurrent survient, on applique ce que l'on appelle des GPO (Group Policy Object). Ce sont un ensemble de règles qui sont appliquées de différentes façons (à tous les ordinateurs, tous les utilisateurs, un groupe, une seule personne, un service ...). C'est ce que l'on appelle une stratégie de groupe. Cela permet de corriger le problème pour tout le monde, et non cas par cas. 
Cependant, le nombre de règles croit avec l'entreprise et les problèmes rencontrés (au CNRS, il y en a plusieurs centaines). Il a donc été mis en place une réunion hebdomadaire (tous les jeudi matin), organisé par Baptiste BARAKOWSKY, responsable du pôle "Gestion des infrastructures", afin d'expliciter ces règles.
\newpage

\subsubsection{Le ticketing}
Un exemple de travail en équipe que j'ai pu observer concerne le principe de ticketing. Une machine virtuelle faisant tourner un serveur OTRS (Open-source Ticket Request System) permet aux utilisateurs qui rencontrent un problème de le notifier au travers d'un ticket d'assistance aux 3 poles du SI. Par la suite, selon la nature du problème rencontré, un certain pôle va prendre en charge ce ticket. Dès l'instant où le ticket est pris en charge, il disparait de l'écran des autres membres du SI, afin de faciliter son traitement et d'éviter les doublons, ce qui ferait perdre beaucoup de temps. Cela n'empeche en rien les autres pôles d'apporter son aide à la personne qui à bloqué le ticket. En effet, toutes les personnes ayant lu le message sont au courant du problème et peuvent apporter leur aide.

\subsubsection{La cohésion de groupe}
La communication et la cohésion de groupe étant très primordiale dans ce genre d'organisation, la "pause café", ou la "pause cigarette" se sont révélées très importantes au sein d'un groupe comme celui-là. En effet, selon les statistiques, 84\% des salariés jugent important d'avoir une machine à café à disposition dans les locaux de l'entreprise. Cela permet une communication accrue entre les membres des differentes équipes, en plus d'être un lieu de détente et d'échanges.  

\subsubsection{Skype Entreprise}
La communication étant très importante dans un groupe comme celui-la, d'autant plus que les bureaux peuvent être relativement éloignés (environ 50 mètres), il peut-être pénible d'avoir à se lever et de mrcher pour passer une information qui ne nécessite pas l'envoie d'un mail officiel. C'est pourquoi l'IFSeM utilise Skype Entreprise, leur permettant de creer des groupes et de discuter, d'envoyer des informations, poser des questions\dots depuis leur bureaux respectifs.
