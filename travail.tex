\subsection{Hotline et assistance aux utilisateurs : MCO-Windows}
Une importante partie du travail consistait en l'aide aux utilisateurs. De ce fait, en plus du système de ticketing évoqué plus haut, le pôle SI faisait aussi office de centre hotline pour les utilisateurs. Tous les membres disposaient d'un téléphone fonctionnant pat téléphonie IP, permettant d'appeler n'importe quel agent du CNRS.

\subsubsection{L'infogérance}
Une fois que le problème est identifié (appel ou ticket), une phase de recherche commence. La première chose à verifier est si ce problème à déjà été reporté et/ou si un problème proche à déjà été corrigé dans le passé. Si c'est le cas, il faudra alors appliqué une GPO. Une GPO est une règle que l'on peut appliqué à différents niveau. Tout cela demande une grande organisation. 
\medbreak
\textit{Un exemple concret que j'ai rencontré lors de mon stage est un problème qui survient lorsqu'une machine uniquement de la marque Hewlett-Packard Company (HP) se connecte à un VPN. Un processus en fond détectait la connexion VPN comme étant une connexion filaire et désactivait donc automatiquement le Wifi, ayant pour effet de couper la connexion avec le VPN. Cette personne ne pouvait don pas travailler de chez elles, car l'ensemble des application nationales ne sont accessibles que depuis l'intérieur du cnrs, donc soit physiquement soit au travers d'un VPN. La solution consistait donc à arrêter ce processus dès le démarrage de toutes les machines HP.}
\medbreak
En plus des GPO qui régissent le parc informatique, l'IFSeM dispose d'un serveur Dell, appelé K1000, permettant de gérer l'ensemble du parc informatique. Ce serveur permet de savoir exactement qui s'est connecté sur quelle machine, à quelle heure, qui est connecté en temps réel\dots De plus, cela permet de gerer les mises à jour système et logiciel pour tout le monde. Le but étant qu'un utilisateur ne se rendent compte de rien. Les mises à jour sont donc éfféctuées avec des options qui les rendent silencieuse, les grosses maj sont efféctuées de nuit\dots 

\newpage

\subsubsection{Déploiement de l'image IFSeM}
En informatique, on parle d'image système ou image, un système d'exploitation (dans notre cas un système Windows 10 Pro que l'on va deployer en grand nombre. On peut stocker une image sur un CD (comme les CD d'installation Windows), une clé USB ou encore accessible sur un serveur depuis le réseau (cas du CNRS) à l'aide du protocole PXE. 
\medbreak
Lorsqu'une nouvelle machine (portable ou fixe) était livré, il fallait déployer dessus l'image standard crée par l'IFSeM. En effet, jusqu'à présent, chaque délégation possédait sa propre image, que le SSI local (service système d'information, chaque délégation dispose d'un SSI local) déployait sur chaque machine qu'ils recevaient. Cependant, depuis quelques mois, il a été décidé de déployer une image unique sur tous les postes de chaque délégation, afin de faciliter le travail et la gestion à distance du parc informatique.
\medbreak
J'ai donc du me déplacer, avec l'équipe Gestion des Infrastructures sur le site de Meudon, afin de déployer des images systèmes et installer les postes qui venaient d'être livrés. Cela permet de mieux gérer l'infrastructure de l'ensemble des postes (En annexe l'ordre de mission confirmé par le service des stages). Cette intervention sur un autre site que celui de Villejuif m'a permis de découvrir un autre environnement de travail et de me rendre compte des difficultés des interventions sur un autre site. En effet, en plus des problèmes réseaux, nous avons du attendre que le matériel demandé arrive car il n'avait pas été préparé en amont par le SSI local. 

\subsection{Maintien des applications nationales : MCO-Linux}
Une des grosses parties du travail que j'ai effectué durant ce stage fut du MCO-Linux. MCO, ou Maintient en condition opérationnelles consiste à assurer une continuité de service et une disponibilité des services hébergés sur les hyperviseurs. \\
La plupart des services sont hébergés sur des hyperviseurs, eux même divisés en plusieurs machines virtuelles. Il faut alors impérativement garantir une accessibilité à ces services. Lors d'un bug, nous devions alors basculer sur le serveur de secours, afin que les utilisateurs ne se rendent pas compte du bug, puis de corriger ce dernier. Cette tache dépendait alors complètement du type de service et du type de bug. Le plus souvent, un simple reboot de la VM (Virtual-Machine), l'observation des logs (journaux tenus à jour par la machine), ou même une restauration à une version antérieure suffisait à résoudre le bug. 

\subsubsection{Sauvegardes des machines virtuelles}
L'IFSeM disposait d'un serveur de stockage (10 To ) permettant de stocker les sauvegardes des différentes applications. Ce serveur (qui est en réalité une VM), fait tourner l'application BackUpPC, une application avec une interface web permettant de gerer les sauvegardes de toutes les VM. La place étant limité, un système de roulement des sauvegardes est mis en place automatiquement par BackUpPC, afin de ne garder que certaines sauvegardes (partielles et totales). \\
Une des tâches que j'ai effectué fut de mettre en place un planning des sauvegardes. Une autre tache que j'ai du effectuer fût d'augmenter la place du serveur de sauvegarde. En effet, plus le temps passe, plus les sauvegardes sont lourdes. Il est donc arrivé à un point où le serveur était plein. On a donc sollicité mon aide afin d'augmenter la taille de la partition du disque virtuelle accessible depuis la VM, sans pour autant écraser les sauvegardes. Après avoir, pendant une journée, reproduis la situation sur un environnement de test, j'ai réussi sans encombre l'agrandissement de la partition, qui est une opération qui peut s'avérer périlleuse, surtout que seule les outils en ligne de commande étaient disponible : Parted, resize ...

\subsection{La documentation}
Une grand partie de mon travail durant ce stage fut de compléter la documentation. En effet, chacun des pôles cités dans 1.3 possédaient leur propre documentation personnel, souvent au format Word ou même bloc-note. Un projet en cours lors de mon arrivé était de regrouper toutes les docs sur un Wiki accessible par tout l'IFSeM. J'ai donc du recopier et transférer les documents depuis les fichiers world vers le Wiki. \\
De plus, afin de vérifier que les informations étaient corrects, il m'a été demandé de suivre et de réaliser le contenu de ces documentations et de prendre des captures d'écran des messages affichés. 
Ce travail m'a pris environ 50\% de mon temps. Bien que peut intéressante, cette tâche est pourtant nécessaire et permet aux employés de gagner beaucoup de temps. 